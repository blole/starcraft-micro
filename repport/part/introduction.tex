\section{Introduction}
RTS games provide a very interesting platform for developing and testing multi agent cooperative behavior.
%In game theoretic terms, RTS games can be considered symmetric, non-zero-sum, simultaneous, imperfect information, combinatorial games.
That which separates RTS games the most from traditional games like chess is the huge branching factor when taking every possible combination of every unit's moves into account, and even in a 10 vs 10 unit battle, all the different ways a battle could play out quickly grows beyond what is feasible to do a exhaustive search on.
On top of that, a complete model of how the game mechanics work might not be available, and new decisions have to be taken in only the time between each frame
\footnote{
As an example, in the 2011 AIIDE Starcraft AI Competition, 55 ms were given to the bots per frame.
\url{https://skatgame.net/mburo/sc2011/rules.html}
}
.

The game of Starcraft is a good platform for evaluating different methods for addressing these difficulties that RTS games pose and several tournaments are held yearly
\footnote{
\url{http://www.sscaitournament.com/} \\
\url{http://www.aiide.org/starcraft} \\
\url{http://bots-stats.krasi0.com/} (ladder)
}
.
